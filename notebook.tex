
% Default to the notebook output style

    


% Inherit from the specified cell style.




    
\documentclass[11pt]{article}

    
    
    \usepackage[T1]{fontenc}
    % Nicer default font (+ math font) than Computer Modern for most use cases
    \usepackage{mathpazo}

    % Basic figure setup, for now with no caption control since it's done
    % automatically by Pandoc (which extracts ![](path) syntax from Markdown).
    \usepackage{graphicx}
    % We will generate all images so they have a width \maxwidth. This means
    % that they will get their normal width if they fit onto the page, but
    % are scaled down if they would overflow the margins.
    \makeatletter
    \def\maxwidth{\ifdim\Gin@nat@width>\linewidth\linewidth
    \else\Gin@nat@width\fi}
    \makeatother
    \let\Oldincludegraphics\includegraphics
    % Set max figure width to be 80% of text width, for now hardcoded.
    \renewcommand{\includegraphics}[1]{\Oldincludegraphics[width=.8\maxwidth]{#1}}
    % Ensure that by default, figures have no caption (until we provide a
    % proper Figure object with a Caption API and a way to capture that
    % in the conversion process - todo).
    \usepackage{caption}
    \DeclareCaptionLabelFormat{nolabel}{}
    \captionsetup{labelformat=nolabel}

    \usepackage{adjustbox} % Used to constrain images to a maximum size 
    \usepackage{xcolor} % Allow colors to be defined
    \usepackage{enumerate} % Needed for markdown enumerations to work
    \usepackage{geometry} % Used to adjust the document margins
    \usepackage{amsmath} % Equations
    \usepackage{amssymb} % Equations
    \usepackage{textcomp} % defines textquotesingle
    % Hack from http://tex.stackexchange.com/a/47451/13684:
    \AtBeginDocument{%
        \def\PYZsq{\textquotesingle}% Upright quotes in Pygmentized code
    }
    \usepackage{upquote} % Upright quotes for verbatim code
    \usepackage{eurosym} % defines \euro
    \usepackage[mathletters]{ucs} % Extended unicode (utf-8) support
    \usepackage[utf8x]{inputenc} % Allow utf-8 characters in the tex document
    \usepackage{fancyvrb} % verbatim replacement that allows latex
    \usepackage{grffile} % extends the file name processing of package graphics 
                         % to support a larger range 
    % The hyperref package gives us a pdf with properly built
    % internal navigation ('pdf bookmarks' for the table of contents,
    % internal cross-reference links, web links for URLs, etc.)
    \usepackage{hyperref}
    \usepackage{longtable} % longtable support required by pandoc >1.10
    \usepackage{booktabs}  % table support for pandoc > 1.12.2
    \usepackage[inline]{enumitem} % IRkernel/repr support (it uses the enumerate* environment)
    \usepackage[normalem]{ulem} % ulem is needed to support strikethroughs (\sout)
                                % normalem makes italics be italics, not underlines
    

    
    
    % Colors for the hyperref package
    \definecolor{urlcolor}{rgb}{0,.145,.698}
    \definecolor{linkcolor}{rgb}{.71,0.21,0.01}
    \definecolor{citecolor}{rgb}{.12,.54,.11}

    % ANSI colors
    \definecolor{ansi-black}{HTML}{3E424D}
    \definecolor{ansi-black-intense}{HTML}{282C36}
    \definecolor{ansi-red}{HTML}{E75C58}
    \definecolor{ansi-red-intense}{HTML}{B22B31}
    \definecolor{ansi-green}{HTML}{00A250}
    \definecolor{ansi-green-intense}{HTML}{007427}
    \definecolor{ansi-yellow}{HTML}{DDB62B}
    \definecolor{ansi-yellow-intense}{HTML}{B27D12}
    \definecolor{ansi-blue}{HTML}{208FFB}
    \definecolor{ansi-blue-intense}{HTML}{0065CA}
    \definecolor{ansi-magenta}{HTML}{D160C4}
    \definecolor{ansi-magenta-intense}{HTML}{A03196}
    \definecolor{ansi-cyan}{HTML}{60C6C8}
    \definecolor{ansi-cyan-intense}{HTML}{258F8F}
    \definecolor{ansi-white}{HTML}{C5C1B4}
    \definecolor{ansi-white-intense}{HTML}{A1A6B2}

    % commands and environments needed by pandoc snippets
    % extracted from the output of `pandoc -s`
    \providecommand{\tightlist}{%
      \setlength{\itemsep}{0pt}\setlength{\parskip}{0pt}}
    \DefineVerbatimEnvironment{Highlighting}{Verbatim}{commandchars=\\\{\}}
    % Add ',fontsize=\small' for more characters per line
    \newenvironment{Shaded}{}{}
    \newcommand{\KeywordTok}[1]{\textcolor[rgb]{0.00,0.44,0.13}{\textbf{{#1}}}}
    \newcommand{\DataTypeTok}[1]{\textcolor[rgb]{0.56,0.13,0.00}{{#1}}}
    \newcommand{\DecValTok}[1]{\textcolor[rgb]{0.25,0.63,0.44}{{#1}}}
    \newcommand{\BaseNTok}[1]{\textcolor[rgb]{0.25,0.63,0.44}{{#1}}}
    \newcommand{\FloatTok}[1]{\textcolor[rgb]{0.25,0.63,0.44}{{#1}}}
    \newcommand{\CharTok}[1]{\textcolor[rgb]{0.25,0.44,0.63}{{#1}}}
    \newcommand{\StringTok}[1]{\textcolor[rgb]{0.25,0.44,0.63}{{#1}}}
    \newcommand{\CommentTok}[1]{\textcolor[rgb]{0.38,0.63,0.69}{\textit{{#1}}}}
    \newcommand{\OtherTok}[1]{\textcolor[rgb]{0.00,0.44,0.13}{{#1}}}
    \newcommand{\AlertTok}[1]{\textcolor[rgb]{1.00,0.00,0.00}{\textbf{{#1}}}}
    \newcommand{\FunctionTok}[1]{\textcolor[rgb]{0.02,0.16,0.49}{{#1}}}
    \newcommand{\RegionMarkerTok}[1]{{#1}}
    \newcommand{\ErrorTok}[1]{\textcolor[rgb]{1.00,0.00,0.00}{\textbf{{#1}}}}
    \newcommand{\NormalTok}[1]{{#1}}
    
    % Additional commands for more recent versions of Pandoc
    \newcommand{\ConstantTok}[1]{\textcolor[rgb]{0.53,0.00,0.00}{{#1}}}
    \newcommand{\SpecialCharTok}[1]{\textcolor[rgb]{0.25,0.44,0.63}{{#1}}}
    \newcommand{\VerbatimStringTok}[1]{\textcolor[rgb]{0.25,0.44,0.63}{{#1}}}
    \newcommand{\SpecialStringTok}[1]{\textcolor[rgb]{0.73,0.40,0.53}{{#1}}}
    \newcommand{\ImportTok}[1]{{#1}}
    \newcommand{\DocumentationTok}[1]{\textcolor[rgb]{0.73,0.13,0.13}{\textit{{#1}}}}
    \newcommand{\AnnotationTok}[1]{\textcolor[rgb]{0.38,0.63,0.69}{\textbf{\textit{{#1}}}}}
    \newcommand{\CommentVarTok}[1]{\textcolor[rgb]{0.38,0.63,0.69}{\textbf{\textit{{#1}}}}}
    \newcommand{\VariableTok}[1]{\textcolor[rgb]{0.10,0.09,0.49}{{#1}}}
    \newcommand{\ControlFlowTok}[1]{\textcolor[rgb]{0.00,0.44,0.13}{\textbf{{#1}}}}
    \newcommand{\OperatorTok}[1]{\textcolor[rgb]{0.40,0.40,0.40}{{#1}}}
    \newcommand{\BuiltInTok}[1]{{#1}}
    \newcommand{\ExtensionTok}[1]{{#1}}
    \newcommand{\PreprocessorTok}[1]{\textcolor[rgb]{0.74,0.48,0.00}{{#1}}}
    \newcommand{\AttributeTok}[1]{\textcolor[rgb]{0.49,0.56,0.16}{{#1}}}
    \newcommand{\InformationTok}[1]{\textcolor[rgb]{0.38,0.63,0.69}{\textbf{\textit{{#1}}}}}
    \newcommand{\WarningTok}[1]{\textcolor[rgb]{0.38,0.63,0.69}{\textbf{\textit{{#1}}}}}
    
    
    % Define a nice break command that doesn't care if a line doesn't already
    % exist.
    \def\br{\hspace*{\fill} \\* }
    % Math Jax compatability definitions
    \def\gt{>}
    \def\lt{<}
    % Document parameters
    \title{Report}
    
    
    

    % Pygments definitions
    
\makeatletter
\def\PY@reset{\let\PY@it=\relax \let\PY@bf=\relax%
    \let\PY@ul=\relax \let\PY@tc=\relax%
    \let\PY@bc=\relax \let\PY@ff=\relax}
\def\PY@tok#1{\csname PY@tok@#1\endcsname}
\def\PY@toks#1+{\ifx\relax#1\empty\else%
    \PY@tok{#1}\expandafter\PY@toks\fi}
\def\PY@do#1{\PY@bc{\PY@tc{\PY@ul{%
    \PY@it{\PY@bf{\PY@ff{#1}}}}}}}
\def\PY#1#2{\PY@reset\PY@toks#1+\relax+\PY@do{#2}}

\expandafter\def\csname PY@tok@w\endcsname{\def\PY@tc##1{\textcolor[rgb]{0.73,0.73,0.73}{##1}}}
\expandafter\def\csname PY@tok@c\endcsname{\let\PY@it=\textit\def\PY@tc##1{\textcolor[rgb]{0.25,0.50,0.50}{##1}}}
\expandafter\def\csname PY@tok@cp\endcsname{\def\PY@tc##1{\textcolor[rgb]{0.74,0.48,0.00}{##1}}}
\expandafter\def\csname PY@tok@k\endcsname{\let\PY@bf=\textbf\def\PY@tc##1{\textcolor[rgb]{0.00,0.50,0.00}{##1}}}
\expandafter\def\csname PY@tok@kp\endcsname{\def\PY@tc##1{\textcolor[rgb]{0.00,0.50,0.00}{##1}}}
\expandafter\def\csname PY@tok@kt\endcsname{\def\PY@tc##1{\textcolor[rgb]{0.69,0.00,0.25}{##1}}}
\expandafter\def\csname PY@tok@o\endcsname{\def\PY@tc##1{\textcolor[rgb]{0.40,0.40,0.40}{##1}}}
\expandafter\def\csname PY@tok@ow\endcsname{\let\PY@bf=\textbf\def\PY@tc##1{\textcolor[rgb]{0.67,0.13,1.00}{##1}}}
\expandafter\def\csname PY@tok@nb\endcsname{\def\PY@tc##1{\textcolor[rgb]{0.00,0.50,0.00}{##1}}}
\expandafter\def\csname PY@tok@nf\endcsname{\def\PY@tc##1{\textcolor[rgb]{0.00,0.00,1.00}{##1}}}
\expandafter\def\csname PY@tok@nc\endcsname{\let\PY@bf=\textbf\def\PY@tc##1{\textcolor[rgb]{0.00,0.00,1.00}{##1}}}
\expandafter\def\csname PY@tok@nn\endcsname{\let\PY@bf=\textbf\def\PY@tc##1{\textcolor[rgb]{0.00,0.00,1.00}{##1}}}
\expandafter\def\csname PY@tok@ne\endcsname{\let\PY@bf=\textbf\def\PY@tc##1{\textcolor[rgb]{0.82,0.25,0.23}{##1}}}
\expandafter\def\csname PY@tok@nv\endcsname{\def\PY@tc##1{\textcolor[rgb]{0.10,0.09,0.49}{##1}}}
\expandafter\def\csname PY@tok@no\endcsname{\def\PY@tc##1{\textcolor[rgb]{0.53,0.00,0.00}{##1}}}
\expandafter\def\csname PY@tok@nl\endcsname{\def\PY@tc##1{\textcolor[rgb]{0.63,0.63,0.00}{##1}}}
\expandafter\def\csname PY@tok@ni\endcsname{\let\PY@bf=\textbf\def\PY@tc##1{\textcolor[rgb]{0.60,0.60,0.60}{##1}}}
\expandafter\def\csname PY@tok@na\endcsname{\def\PY@tc##1{\textcolor[rgb]{0.49,0.56,0.16}{##1}}}
\expandafter\def\csname PY@tok@nt\endcsname{\let\PY@bf=\textbf\def\PY@tc##1{\textcolor[rgb]{0.00,0.50,0.00}{##1}}}
\expandafter\def\csname PY@tok@nd\endcsname{\def\PY@tc##1{\textcolor[rgb]{0.67,0.13,1.00}{##1}}}
\expandafter\def\csname PY@tok@s\endcsname{\def\PY@tc##1{\textcolor[rgb]{0.73,0.13,0.13}{##1}}}
\expandafter\def\csname PY@tok@sd\endcsname{\let\PY@it=\textit\def\PY@tc##1{\textcolor[rgb]{0.73,0.13,0.13}{##1}}}
\expandafter\def\csname PY@tok@si\endcsname{\let\PY@bf=\textbf\def\PY@tc##1{\textcolor[rgb]{0.73,0.40,0.53}{##1}}}
\expandafter\def\csname PY@tok@se\endcsname{\let\PY@bf=\textbf\def\PY@tc##1{\textcolor[rgb]{0.73,0.40,0.13}{##1}}}
\expandafter\def\csname PY@tok@sr\endcsname{\def\PY@tc##1{\textcolor[rgb]{0.73,0.40,0.53}{##1}}}
\expandafter\def\csname PY@tok@ss\endcsname{\def\PY@tc##1{\textcolor[rgb]{0.10,0.09,0.49}{##1}}}
\expandafter\def\csname PY@tok@sx\endcsname{\def\PY@tc##1{\textcolor[rgb]{0.00,0.50,0.00}{##1}}}
\expandafter\def\csname PY@tok@m\endcsname{\def\PY@tc##1{\textcolor[rgb]{0.40,0.40,0.40}{##1}}}
\expandafter\def\csname PY@tok@gh\endcsname{\let\PY@bf=\textbf\def\PY@tc##1{\textcolor[rgb]{0.00,0.00,0.50}{##1}}}
\expandafter\def\csname PY@tok@gu\endcsname{\let\PY@bf=\textbf\def\PY@tc##1{\textcolor[rgb]{0.50,0.00,0.50}{##1}}}
\expandafter\def\csname PY@tok@gd\endcsname{\def\PY@tc##1{\textcolor[rgb]{0.63,0.00,0.00}{##1}}}
\expandafter\def\csname PY@tok@gi\endcsname{\def\PY@tc##1{\textcolor[rgb]{0.00,0.63,0.00}{##1}}}
\expandafter\def\csname PY@tok@gr\endcsname{\def\PY@tc##1{\textcolor[rgb]{1.00,0.00,0.00}{##1}}}
\expandafter\def\csname PY@tok@ge\endcsname{\let\PY@it=\textit}
\expandafter\def\csname PY@tok@gs\endcsname{\let\PY@bf=\textbf}
\expandafter\def\csname PY@tok@gp\endcsname{\let\PY@bf=\textbf\def\PY@tc##1{\textcolor[rgb]{0.00,0.00,0.50}{##1}}}
\expandafter\def\csname PY@tok@go\endcsname{\def\PY@tc##1{\textcolor[rgb]{0.53,0.53,0.53}{##1}}}
\expandafter\def\csname PY@tok@gt\endcsname{\def\PY@tc##1{\textcolor[rgb]{0.00,0.27,0.87}{##1}}}
\expandafter\def\csname PY@tok@err\endcsname{\def\PY@bc##1{\setlength{\fboxsep}{0pt}\fcolorbox[rgb]{1.00,0.00,0.00}{1,1,1}{\strut ##1}}}
\expandafter\def\csname PY@tok@kc\endcsname{\let\PY@bf=\textbf\def\PY@tc##1{\textcolor[rgb]{0.00,0.50,0.00}{##1}}}
\expandafter\def\csname PY@tok@kd\endcsname{\let\PY@bf=\textbf\def\PY@tc##1{\textcolor[rgb]{0.00,0.50,0.00}{##1}}}
\expandafter\def\csname PY@tok@kn\endcsname{\let\PY@bf=\textbf\def\PY@tc##1{\textcolor[rgb]{0.00,0.50,0.00}{##1}}}
\expandafter\def\csname PY@tok@kr\endcsname{\let\PY@bf=\textbf\def\PY@tc##1{\textcolor[rgb]{0.00,0.50,0.00}{##1}}}
\expandafter\def\csname PY@tok@bp\endcsname{\def\PY@tc##1{\textcolor[rgb]{0.00,0.50,0.00}{##1}}}
\expandafter\def\csname PY@tok@fm\endcsname{\def\PY@tc##1{\textcolor[rgb]{0.00,0.00,1.00}{##1}}}
\expandafter\def\csname PY@tok@vc\endcsname{\def\PY@tc##1{\textcolor[rgb]{0.10,0.09,0.49}{##1}}}
\expandafter\def\csname PY@tok@vg\endcsname{\def\PY@tc##1{\textcolor[rgb]{0.10,0.09,0.49}{##1}}}
\expandafter\def\csname PY@tok@vi\endcsname{\def\PY@tc##1{\textcolor[rgb]{0.10,0.09,0.49}{##1}}}
\expandafter\def\csname PY@tok@vm\endcsname{\def\PY@tc##1{\textcolor[rgb]{0.10,0.09,0.49}{##1}}}
\expandafter\def\csname PY@tok@sa\endcsname{\def\PY@tc##1{\textcolor[rgb]{0.73,0.13,0.13}{##1}}}
\expandafter\def\csname PY@tok@sb\endcsname{\def\PY@tc##1{\textcolor[rgb]{0.73,0.13,0.13}{##1}}}
\expandafter\def\csname PY@tok@sc\endcsname{\def\PY@tc##1{\textcolor[rgb]{0.73,0.13,0.13}{##1}}}
\expandafter\def\csname PY@tok@dl\endcsname{\def\PY@tc##1{\textcolor[rgb]{0.73,0.13,0.13}{##1}}}
\expandafter\def\csname PY@tok@s2\endcsname{\def\PY@tc##1{\textcolor[rgb]{0.73,0.13,0.13}{##1}}}
\expandafter\def\csname PY@tok@sh\endcsname{\def\PY@tc##1{\textcolor[rgb]{0.73,0.13,0.13}{##1}}}
\expandafter\def\csname PY@tok@s1\endcsname{\def\PY@tc##1{\textcolor[rgb]{0.73,0.13,0.13}{##1}}}
\expandafter\def\csname PY@tok@mb\endcsname{\def\PY@tc##1{\textcolor[rgb]{0.40,0.40,0.40}{##1}}}
\expandafter\def\csname PY@tok@mf\endcsname{\def\PY@tc##1{\textcolor[rgb]{0.40,0.40,0.40}{##1}}}
\expandafter\def\csname PY@tok@mh\endcsname{\def\PY@tc##1{\textcolor[rgb]{0.40,0.40,0.40}{##1}}}
\expandafter\def\csname PY@tok@mi\endcsname{\def\PY@tc##1{\textcolor[rgb]{0.40,0.40,0.40}{##1}}}
\expandafter\def\csname PY@tok@il\endcsname{\def\PY@tc##1{\textcolor[rgb]{0.40,0.40,0.40}{##1}}}
\expandafter\def\csname PY@tok@mo\endcsname{\def\PY@tc##1{\textcolor[rgb]{0.40,0.40,0.40}{##1}}}
\expandafter\def\csname PY@tok@ch\endcsname{\let\PY@it=\textit\def\PY@tc##1{\textcolor[rgb]{0.25,0.50,0.50}{##1}}}
\expandafter\def\csname PY@tok@cm\endcsname{\let\PY@it=\textit\def\PY@tc##1{\textcolor[rgb]{0.25,0.50,0.50}{##1}}}
\expandafter\def\csname PY@tok@cpf\endcsname{\let\PY@it=\textit\def\PY@tc##1{\textcolor[rgb]{0.25,0.50,0.50}{##1}}}
\expandafter\def\csname PY@tok@c1\endcsname{\let\PY@it=\textit\def\PY@tc##1{\textcolor[rgb]{0.25,0.50,0.50}{##1}}}
\expandafter\def\csname PY@tok@cs\endcsname{\let\PY@it=\textit\def\PY@tc##1{\textcolor[rgb]{0.25,0.50,0.50}{##1}}}

\def\PYZbs{\char`\\}
\def\PYZus{\char`\_}
\def\PYZob{\char`\{}
\def\PYZcb{\char`\}}
\def\PYZca{\char`\^}
\def\PYZam{\char`\&}
\def\PYZlt{\char`\<}
\def\PYZgt{\char`\>}
\def\PYZsh{\char`\#}
\def\PYZpc{\char`\%}
\def\PYZdl{\char`\$}
\def\PYZhy{\char`\-}
\def\PYZsq{\char`\'}
\def\PYZdq{\char`\"}
\def\PYZti{\char`\~}
% for compatibility with earlier versions
\def\PYZat{@}
\def\PYZlb{[}
\def\PYZrb{]}
\makeatother


    % Exact colors from NB
    \definecolor{incolor}{rgb}{0.0, 0.0, 0.5}
    \definecolor{outcolor}{rgb}{0.545, 0.0, 0.0}



    
    % Prevent overflowing lines due to hard-to-break entities
    \sloppy 
    % Setup hyperref package
    \hypersetup{
      breaklinks=true,  % so long urls are correctly broken across lines
      colorlinks=true,
      urlcolor=urlcolor,
      linkcolor=linkcolor,
      citecolor=citecolor,
      }
    % Slightly bigger margins than the latex defaults
    
    \geometry{verbose,tmargin=1in,bmargin=1in,lmargin=1in,rmargin=1in}
    
    

    \begin{document}
    
    
    \maketitle
    
    

    
    \section{Simple implementation of a Random Forest for classification in
C
++}\label{simple-implementation-of-a-random-forest-for-classification-in-c}

\subsection{Raúl Mar}\label{rauxfal-mar}

\subsection{A00512318}\label{a00512318}

    \paragraph{What is a Random Forest?}\label{what-is-a-random-forest}

Random Forest is an ensemble algorithm for supervised classification and
regression. The general method for Random Forests was first proposed by
Tim Kam Ho in 1995, an extension of the algorithm was made later by Leo
Breiman in 2001. Essentially, they are a modification of bagging that
builds a large collection of of de-correlated trees, and then averages
them. The main idea is to have a combination of weak tree predictors
such that the features selected to split is chosen independently and
with the same distribution for each of the trees in the forest in order
to create a strong classifier.

\begin{figure}
\centering
\includegraphics{https://cdn-images-1.medium.com/max/1600/0*tG-IWcxL1jg7RkT0.png}
\caption{Random Forest Image}
\end{figure}

\subparagraph{Algorithm}\label{algorithm}

\begin{enumerate}
\def\labelenumi{\arabic{enumi}.}
\tightlist
\item
  For 1 to B where B is the number of trees to build in the forest
\end{enumerate}

\begin{itemize}
\tightlist
\item
  Draw a bootstrap sample z of size n from the Data set.
\item
  Grow each tree by:

  \begin{itemize}
  \tightlist
  \item
    Selecting m features at random from the M features in the sample z
  \item
    Pick the best split point among the m
  \item
    Split the node into two children nodes
  \item
    Stop if we reached a pure node or if we reached the maximum depth
    for the tree
  \end{itemize}
\end{itemize}

\begin{enumerate}
\def\labelenumi{\arabic{enumi}.}
\setcounter{enumi}{1}
\tightlist
\item
  Output the ensemble of the trees
\end{enumerate}

The number of trees to create depends on the dataset and the user
criteria. The m features to select is normally sqrt(lenght of total
features in sample). The maximun depth for the tree should be at least
\href{https://web.stanford.edu/~hastie/Papers/ESLII.pdf}{2}.

The way in which we split the nodes can be done by calculating the gini
index or the information gain. For this project I chose to split by
calculating the gini index mainly because for the Laboratory 3 about
Decision Trees, we implemented the information gain, so I wanted to try
the other method to understand it better.

\subparagraph{Gini Index}\label{gini-index}

For computing the gini index, what we do is:

\begin{enumerate}
\def\labelenumi{\arabic{enumi}.}
\item
  For all the splitting features m we chose the one that gives the
  maximum gini coeficient by:
\item
  We calculate for each of the attributes in feature m we pick the
  attribute that gives the highest gini coeficient using this formula:
\end{enumerate}

\begin{align}
\ Gini = 1- \sum\limits_{i=1}^K (P_k)^2  \\
Where \ K \ is \ the \ set \ of \ all \ the \ attributes \ for \ a \ feature \ m.
\end{align}

\begin{enumerate}
\def\labelenumi{\arabic{enumi}.}
\setcounter{enumi}{2}
\tightlist
\item
  We split by the feature m that gave the highest gini coeficient
\end{enumerate}

    \paragraph{What I did}\label{what-i-did}

The main motivation for implementing from scratch a random forests in
C++ was that I believe that by doing everything by myself without the
use of any library, I can obtain a deeper understanding of how the
algorithm works. I want to clarify that the implementation that I did
does not cover the full functionalities of a Random Forest. My
implementation is capable of:

\begin{itemize}
\tightlist
\item
  Building n trees in parallel each with sqrt(m) features where M is the
  total number of features from the dataset.
\item
  Grow each tree by using the gini index as the splitting criterion
\item
  Stop splitting if we reach a pure node, or if the have reached the
  max\_depth provided by the user.
\item
  Classify a query by querying each tree in parallel.
\end{itemize}

For it to be a full implementation of a Random Forest I would need to
support: * Feature importance * OOB error * Regression * Support
datasets with missing values

\subparagraph{Dataset}\label{dataset}

The dataset that I used in order to train/test and compare against
scikit-learn implementation of Random Forests was
\href{https://www.kaggle.com/iabhishekofficial/mobile-price-classification/data}{this}.
The data set consists on 2000 records which describe the specifications
of a mobile phone and classifies it in a price range:

\begin{itemize}
\tightlist
\item
  0(low cost)
\item
  1(medium cost)
\item
  2(high cost)
\item
  3(very high cost)
\end{itemize}

The dataset was divided into 70\%(1400 samples) for training and
30\%(600 samples) for testing. The reason that I chose to split it in
that way was because that is what I
\href{https://www.kaggle.com/general/9456}{found} as the "normal" thing
to to for random forests classifiers.

    \paragraph{Results}\label{results}

    In order to have some way of comparing my implementation, I opted for
comparing against scikit-learn
\href{http://scikit-learn.org/stable/modules/generated/sklearn.ensemble.RandomForestClassifier.html\#sklearn.ensemble.RandomForestClassifier}{implementation}
of Random Forests Classifier, and the reason of that is because I have
already used it in the past with different algorithms for different
projects and I found it really easy to use and very well documented.
Therefore, my aim is to create a similar implementation that works a lot
like the one from scikit-learn.

The tests that I performed where to measure building time, predicting
time and score. I will show some graphics where we can see how miserable
I did.

    \begin{Verbatim}[commandchars=\\\{\}]
{\color{incolor}In [{\color{incolor}1}]:} \PY{k+kn}{import} \PY{n+nn}{numpy} \PY{k}{as} \PY{n+nn}{np}
        
        \PY{l+s+sd}{\PYZdq{}\PYZdq{}\PYZdq{}}
        \PY{l+s+sd}{This first batch of tests consist of building 10 forests. Each with first 1 }
        \PY{l+s+sd}{tree, then 2, 3,......10 each with max depth of 2. The main intention of this }
        \PY{l+s+sd}{was to see how accurate can a random forest be with small and short trees.}
        \PY{l+s+sd}{\PYZdq{}\PYZdq{}\PYZdq{}}
        
        \PY{c+c1}{\PYZsh{} results for building forest with trees from 1 10 with max\PYZhy{}depth = 2}
        \PY{n}{scikit\PYZus{}build\PYZus{}time} \PY{o}{=} \PY{n}{np}\PY{o}{.}\PY{n}{array}\PY{p}{(}\PY{p}{[}\PY{p}{(}\PY{l+m+mi}{1}\PY{p}{,} \PY{l+m+mf}{0.11239409446716309}\PY{p}{)}\PY{p}{,} \PY{p}{(}\PY{l+m+mi}{2}\PY{p}{,} \PY{l+m+mf}{0.11244893074035645}\PY{p}{)}\PY{p}{,} \PY{p}{(}\PY{l+m+mi}{3}\PY{p}{,} \PY{l+m+mf}{0.1103048324584961}\PY{p}{)}\PY{p}{,} \PY{p}{(}\PY{l+m+mi}{4}\PY{p}{,} \PY{l+m+mf}{0.11242198944091797}\PY{p}{)}\PY{p}{,} \PY{p}{(}\PY{l+m+mi}{5}\PY{p}{,} \PY{l+m+mf}{0.11184287071228027}\PY{p}{)}\PY{p}{,} \PY{p}{(}\PY{l+m+mi}{6}\PY{p}{,} \PY{l+m+mf}{0.11103200912475586}\PY{p}{)}\PY{p}{,} \PY{p}{(}\PY{l+m+mi}{7}\PY{p}{,} \PY{l+m+mf}{0.11551380157470703}\PY{p}{)}\PY{p}{,} \PY{p}{(}\PY{l+m+mi}{8}\PY{p}{,} \PY{l+m+mf}{0.11576175689697266}\PY{p}{)}\PY{p}{,} \PY{p}{(}\PY{l+m+mi}{9}\PY{p}{,} \PY{l+m+mf}{0.11564993858337402}\PY{p}{)}\PY{p}{,} \PY{p}{(}\PY{l+m+mi}{10}\PY{p}{,} \PY{l+m+mf}{0.1198725700378418}\PY{p}{)}\PY{p}{]}\PY{p}{)}
        \PY{n}{my\PYZus{}build\PYZus{}time} \PY{o}{=} \PY{n}{np}\PY{o}{.}\PY{n}{array}\PY{p}{(}\PY{p}{[}\PY{p}{(}\PY{l+m+mi}{1}\PY{p}{,} \PY{l+m+mf}{1.567}\PY{p}{)}\PY{p}{,} \PY{p}{(}\PY{l+m+mi}{2}\PY{p}{,} \PY{l+m+mf}{1.447}\PY{p}{)}\PY{p}{,} \PY{p}{(}\PY{l+m+mi}{3}\PY{p}{,} \PY{l+m+mf}{1.539}\PY{p}{)}\PY{p}{,} \PY{p}{(}\PY{l+m+mi}{4}\PY{p}{,} \PY{l+m+mf}{2.149}\PY{p}{)}\PY{p}{,} \PY{p}{(}\PY{l+m+mi}{5}\PY{p}{,} \PY{l+m+mf}{2.911}\PY{p}{)}\PY{p}{,} \PY{p}{(}\PY{l+m+mi}{6}\PY{p}{,} \PY{l+m+mf}{3.708}\PY{p}{)}\PY{p}{,} \PY{p}{(}\PY{l+m+mi}{7}\PY{p}{,} \PY{l+m+mf}{8.441}\PY{p}{)}\PY{p}{,} \PY{p}{(}\PY{l+m+mi}{8}\PY{p}{,} \PY{l+m+mf}{7.846}\PY{p}{)}\PY{p}{,} \PY{p}{(}\PY{l+m+mi}{9}\PY{p}{,} \PY{l+m+mf}{13.001}\PY{p}{)}\PY{p}{,} \PY{p}{(}\PY{l+m+mi}{10}\PY{p}{,} \PY{l+m+mf}{10.557}\PY{p}{)}\PY{p}{]}\PY{p}{)}
        
        \PY{c+c1}{\PYZsh{} results for predicting forest with the same trees }
        \PY{n}{sckikit\PYZus{}predict\PYZus{}time} \PY{o}{=} \PY{n}{np}\PY{o}{.}\PY{n}{array}\PY{p}{(}\PY{p}{[}\PY{p}{(}\PY{l+m+mi}{1}\PY{p}{,} \PY{l+m+mf}{0.2633333333333333}\PY{p}{)}\PY{p}{,} \PY{p}{(}\PY{l+m+mi}{2}\PY{p}{,} \PY{l+m+mf}{0.6066666666666667}\PY{p}{)}\PY{p}{,} \PY{p}{(}\PY{l+m+mi}{3}\PY{p}{,} \PY{l+m+mf}{0.495}\PY{p}{)}\PY{p}{,} \PY{p}{(}\PY{l+m+mi}{4}\PY{p}{,} \PY{l+m+mf}{0.7766666666666666}\PY{p}{)}\PY{p}{,} \PY{p}{(}\PY{l+m+mi}{5}\PY{p}{,} \PY{l+m+mf}{0.5016666666666667}\PY{p}{)}\PY{p}{,} \PY{p}{(}\PY{l+m+mi}{6}\PY{p}{,} \PY{l+m+mf}{0.6366666666666667}\PY{p}{)}\PY{p}{,} \PY{p}{(}\PY{l+m+mi}{7}\PY{p}{,} \PY{l+m+mf}{0.5516666666666666}\PY{p}{)}\PY{p}{,} \PY{p}{(}\PY{l+m+mi}{8}\PY{p}{,} \PY{l+m+mf}{0.5116666666666667}\PY{p}{)}\PY{p}{,} \PY{p}{(}\PY{l+m+mi}{9}\PY{p}{,} \PY{l+m+mf}{0.545}\PY{p}{)}\PY{p}{,} \PY{p}{(}\PY{l+m+mi}{10}\PY{p}{,} \PY{l+m+mf}{0.77}\PY{p}{)}\PY{p}{]}\PY{p}{)}
        \PY{n}{my\PYZus{}predict\PYZus{}time} \PY{o}{=} \PY{n}{np}\PY{o}{.}\PY{n}{array}\PY{p}{(}\PY{p}{[}\PY{p}{(}\PY{l+m+mi}{1}\PY{p}{,} \PY{l+m+mi}{0}\PY{p}{)}\PY{p}{,} \PY{p}{(}\PY{l+m+mi}{2}\PY{p}{,} \PY{l+m+mf}{0.069}\PY{p}{)}\PY{p}{,} \PY{p}{(}\PY{l+m+mi}{3}\PY{p}{,} \PY{l+m+mf}{0.072}\PY{p}{)}\PY{p}{,} \PY{p}{(}\PY{l+m+mi}{4}\PY{p}{,} \PY{l+m+mf}{0.074}\PY{p}{)}\PY{p}{,} \PY{p}{(}\PY{l+m+mi}{5}\PY{p}{,} \PY{l+m+mf}{0.072}\PY{p}{)}\PY{p}{,} \PY{p}{(}\PY{l+m+mi}{6}\PY{p}{,} \PY{l+m+mf}{0.083}\PY{p}{)}\PY{p}{,} \PY{p}{(}\PY{l+m+mi}{7}\PY{p}{,} \PY{l+m+mf}{0.073}\PY{p}{)}\PY{p}{,} \PY{p}{(}\PY{l+m+mi}{8}\PY{p}{,} \PY{l+m+mf}{0.076}\PY{p}{)}\PY{p}{,} \PY{p}{(}\PY{l+m+mi}{9}\PY{p}{,} \PY{l+m+mf}{0.084}\PY{p}{)}\PY{p}{,} \PY{p}{(}\PY{l+m+mi}{10}\PY{p}{,} \PY{l+m+mf}{0.074}\PY{p}{)}\PY{p}{]}\PY{p}{)}
        
        \PY{c+c1}{\PYZsh{} score from previous trees}
        \PY{n}{sckikit\PYZus{}score} \PY{o}{=} \PY{n}{np}\PY{o}{.}\PY{n}{array}\PY{p}{(}\PY{p}{[}\PY{p}{(}\PY{l+m+mi}{1}\PY{p}{,} \PY{l+m+mf}{0.49}\PY{p}{)}\PY{p}{,}\PY{p}{(}\PY{l+m+mi}{2}\PY{p}{,} \PY{l+m+mf}{0.47}\PY{p}{)}\PY{p}{,}\PY{p}{(}\PY{l+m+mi}{3}\PY{p}{,} \PY{l+m+mf}{0.5116666666666667}\PY{p}{)}\PY{p}{,}\PY{p}{(}\PY{l+m+mi}{4}\PY{p}{,} \PY{l+m+mf}{0.545}\PY{p}{)}\PY{p}{,}\PY{p}{(}\PY{l+m+mi}{5}\PY{p}{,} \PY{l+m+mf}{0.52}\PY{p}{)}\PY{p}{,}\PY{p}{(}\PY{l+m+mi}{6}\PY{p}{,} \PY{l+m+mf}{0.5683333333333334}\PY{p}{)}\PY{p}{,}\PY{p}{(}\PY{l+m+mi}{7}\PY{p}{,} \PY{l+m+mf}{0.7066666666666667}\PY{p}{)}\PY{p}{,}\PY{p}{(}\PY{l+m+mi}{8}\PY{p}{,} \PY{l+m+mf}{0.58}\PY{p}{)}\PY{p}{,}\PY{p}{(}\PY{l+m+mi}{9}\PY{p}{,} \PY{l+m+mf}{0.5266666666666666}\PY{p}{)}\PY{p}{,}\PY{p}{(}\PY{l+m+mi}{10}\PY{p}{,} \PY{l+m+mf}{0.695}\PY{p}{)}\PY{p}{]}\PY{p}{)}
        \PY{n}{my\PYZus{}score} \PY{o}{=} \PY{n}{np}\PY{o}{.}\PY{n}{array}\PY{p}{(}\PY{p}{[}\PY{p}{(}\PY{l+m+mi}{1}\PY{p}{,} \PY{l+m+mf}{0.256667}\PY{p}{)}\PY{p}{,} \PY{p}{(}\PY{l+m+mi}{2}\PY{p}{,} \PY{l+m+mf}{0.815}\PY{p}{)}\PY{p}{,} \PY{p}{(}\PY{l+m+mi}{3}\PY{p}{,} \PY{l+m+mf}{0.805}\PY{p}{)}\PY{p}{,}\PY{p}{(}\PY{l+m+mi}{4}\PY{p}{,} \PY{l+m+mf}{0.813333}\PY{p}{)}\PY{p}{,} \PY{p}{(}\PY{l+m+mi}{5}\PY{p}{,} \PY{l+m+mf}{0.816667}\PY{p}{)}\PY{p}{,} \PY{p}{(}\PY{l+m+mi}{6}\PY{p}{,} \PY{l+m+mf}{0.728333}\PY{p}{)}\PY{p}{,} \PY{p}{(}\PY{l+m+mi}{7}\PY{p}{,} \PY{l+m+mf}{0.823333}\PY{p}{)}\PY{p}{,} \PY{p}{(}\PY{l+m+mi}{8}\PY{p}{,} \PY{l+m+mf}{0.253333}\PY{p}{)}\PY{p}{,} \PY{p}{(}\PY{l+m+mi}{9}\PY{p}{,} \PY{l+m+mf}{0.821667}\PY{p}{)}\PY{p}{,} \PY{p}{(}\PY{l+m+mi}{10}\PY{p}{,} \PY{l+m+mf}{0.263333}\PY{p}{)}\PY{p}{]}\PY{p}{)}
\end{Verbatim}


    \begin{Verbatim}[commandchars=\\\{\}]
{\color{incolor}In [{\color{incolor}9}]:} \PY{k+kn}{import} \PY{n+nn}{matplotlib}\PY{n+nn}{.}\PY{n+nn}{pyplot} \PY{k}{as} \PY{n+nn}{plt}
        \PY{k+kn}{import} \PY{n+nn}{matplotlib}
        
        
        \PY{n}{matplotlib}\PY{o}{.}\PY{n}{rcParams}\PY{p}{[}\PY{l+s+s1}{\PYZsq{}}\PY{l+s+s1}{figure.figsize}\PY{l+s+s1}{\PYZsq{}}\PY{p}{]} \PY{o}{=} \PY{p}{[}\PY{l+m+mi}{15}\PY{p}{,} \PY{l+m+mi}{7}\PY{p}{]}
        \PY{n}{plt}\PY{o}{.}\PY{n}{ylabel}\PY{p}{(}\PY{l+s+s1}{\PYZsq{}}\PY{l+s+s1}{Time in seconds}\PY{l+s+s1}{\PYZsq{}}\PY{p}{)}
        \PY{n}{plt}\PY{o}{.}\PY{n}{xlabel}\PY{p}{(}\PY{l+s+s1}{\PYZsq{}}\PY{l+s+s1}{Number of trees}\PY{l+s+s1}{\PYZsq{}}\PY{p}{)}
        \PY{n}{plt}\PY{o}{.}\PY{n}{title}\PY{p}{(}\PY{l+s+s1}{\PYZsq{}}\PY{l+s+s1}{Time building forest with max\PYZhy{}depth 2 \PYZhy{} figure 1.0\PYZhy{}}\PY{l+s+s1}{\PYZsq{}}\PY{p}{)}
        \PY{n}{plt}\PY{o}{.}\PY{n}{plot}\PY{p}{(}\PY{o}{*}\PY{n+nb}{zip}\PY{p}{(}\PY{o}{*}\PY{n}{scikit\PYZus{}build\PYZus{}time}\PY{p}{)}\PY{p}{,} \PY{n}{label}\PY{o}{=}\PY{l+s+s2}{\PYZdq{}}\PY{l+s+s2}{scikit\PYZhy{}learn}\PY{l+s+s2}{\PYZdq{}}\PY{p}{)}
        \PY{n}{plt}\PY{o}{.}\PY{n}{plot}\PY{p}{(}\PY{o}{*}\PY{n+nb}{zip}\PY{p}{(}\PY{o}{*}\PY{n}{my\PYZus{}build\PYZus{}time}\PY{p}{)}\PY{p}{,} \PY{n}{label}\PY{o}{=}\PY{l+s+s2}{\PYZdq{}}\PY{l+s+s2}{my\PYZus{}implementation}\PY{l+s+s2}{\PYZdq{}}\PY{p}{)}
        \PY{n}{plt}\PY{o}{.}\PY{n}{legend}\PY{p}{(}\PY{p}{)}
\end{Verbatim}


\begin{Verbatim}[commandchars=\\\{\}]
{\color{outcolor}Out[{\color{outcolor}9}]:} <matplotlib.legend.Legend at 0x118b43e48>
\end{Verbatim}
            
    \begin{center}
    \adjustimage{max size={0.9\linewidth}{0.9\paperheight}}{output_6_1.png}
    \end{center}
    { \hspace*{\fill} \\}
    
    In the graph above, we can see how slow my implementation is when
building each forest. The main reason of this is because of the way in
which I am computing the split criterion. The way in which I'm doing it
takes O(n**2) time complexity and space complexity O(n). I later found a
post on
\href{https://stats.stackexchange.com/questions/105487/the-efficiency-of-decision-tree}{post}
on stackexchange that talks about how to improve the performance of a
Random Forest. In that post they explain a O(n) time complexity
algorithm for calculating the splitting criterion. Due to the limited
amount of time I had and all the bugs that I had while coding this
project, I was unable to implement the faster way of building the trees.

    \begin{Verbatim}[commandchars=\\\{\}]
{\color{incolor}In [{\color{incolor}10}]:} \PY{n}{plt}\PY{o}{.}\PY{n}{ylabel}\PY{p}{(}\PY{l+s+s1}{\PYZsq{}}\PY{l+s+s1}{Time in seconds}\PY{l+s+s1}{\PYZsq{}}\PY{p}{)}
         \PY{n}{plt}\PY{o}{.}\PY{n}{xlabel}\PY{p}{(}\PY{l+s+s1}{\PYZsq{}}\PY{l+s+s1}{Number of trees}\PY{l+s+s1}{\PYZsq{}}\PY{p}{)}
         \PY{n}{plt}\PY{o}{.}\PY{n}{title}\PY{p}{(}\PY{l+s+s1}{\PYZsq{}}\PY{l+s+s1}{Time predicting forest with max\PYZhy{}depth 2 \PYZhy{} figure 1.2\PYZhy{}}\PY{l+s+s1}{\PYZsq{}}\PY{p}{)}
         \PY{n}{plt}\PY{o}{.}\PY{n}{plot}\PY{p}{(}\PY{o}{*}\PY{n+nb}{zip}\PY{p}{(}\PY{o}{*}\PY{n}{sckikit\PYZus{}predict\PYZus{}time}\PY{p}{)}\PY{p}{,} \PY{n}{label}\PY{o}{=}\PY{l+s+s2}{\PYZdq{}}\PY{l+s+s2}{scikit\PYZhy{}learn}\PY{l+s+s2}{\PYZdq{}}\PY{p}{)}
         \PY{n}{plt}\PY{o}{.}\PY{n}{plot}\PY{p}{(}\PY{o}{*}\PY{n+nb}{zip}\PY{p}{(}\PY{o}{*}\PY{n}{my\PYZus{}predict\PYZus{}time}\PY{p}{)}\PY{p}{,} \PY{n}{label}\PY{o}{=}\PY{l+s+s2}{\PYZdq{}}\PY{l+s+s2}{my\PYZus{}implementation}\PY{l+s+s2}{\PYZdq{}}\PY{p}{)}
         \PY{n}{plt}\PY{o}{.}\PY{n}{legend}\PY{p}{(}\PY{p}{)}
\end{Verbatim}


\begin{Verbatim}[commandchars=\\\{\}]
{\color{outcolor}Out[{\color{outcolor}10}]:} <matplotlib.legend.Legend at 0x118fafc18>
\end{Verbatim}
            
    \begin{center}
    \adjustimage{max size={0.9\linewidth}{0.9\paperheight}}{output_8_1.png}
    \end{center}
    { \hspace*{\fill} \\}
    
    In the figure above, we can see that for predicting, my implementation
is faster, this is expected since Python is
\href{https://www.quora.com/What-is-Python-written-in}{written in C},
which leads to another conclusion that maybe not only the way in which
I'm calculating the splitting criterion is slow, I think I have some
very bad problems with memory management. Unfortunately I was unable to
prove this though.

    \begin{Verbatim}[commandchars=\\\{\}]
{\color{incolor}In [{\color{incolor}11}]:} \PY{n}{plt}\PY{o}{.}\PY{n}{ylabel}\PY{p}{(}\PY{l+s+s1}{\PYZsq{}}\PY{l+s+s1}{Score}\PY{l+s+s1}{\PYZsq{}}\PY{p}{)}
         \PY{n}{plt}\PY{o}{.}\PY{n}{xlabel}\PY{p}{(}\PY{l+s+s1}{\PYZsq{}}\PY{l+s+s1}{Number of trees}\PY{l+s+s1}{\PYZsq{}}\PY{p}{)}
         \PY{n}{plt}\PY{o}{.}\PY{n}{title}\PY{p}{(}\PY{l+s+s1}{\PYZsq{}}\PY{l+s+s1}{Score from predicting forest with max\PYZhy{}depth 2}\PY{l+s+s1}{\PYZsq{}}\PY{p}{)}
         \PY{n}{plt}\PY{o}{.}\PY{n}{plot}\PY{p}{(}\PY{o}{*}\PY{n+nb}{zip}\PY{p}{(}\PY{o}{*}\PY{n}{sckikit\PYZus{}score}\PY{p}{)}\PY{p}{,} \PY{n}{label}\PY{o}{=}\PY{l+s+s2}{\PYZdq{}}\PY{l+s+s2}{scikit\PYZhy{}learn}\PY{l+s+s2}{\PYZdq{}}\PY{p}{)}
         \PY{n}{plt}\PY{o}{.}\PY{n}{plot}\PY{p}{(}\PY{o}{*}\PY{n+nb}{zip}\PY{p}{(}\PY{o}{*}\PY{n}{my\PYZus{}score}\PY{p}{)}\PY{p}{,} \PY{n}{label}\PY{o}{=}\PY{l+s+s2}{\PYZdq{}}\PY{l+s+s2}{my\PYZus{}implementation}\PY{l+s+s2}{\PYZdq{}}\PY{p}{)}
         \PY{n}{plt}\PY{o}{.}\PY{n}{legend}\PY{p}{(}\PY{p}{)}
\end{Verbatim}


\begin{Verbatim}[commandchars=\\\{\}]
{\color{outcolor}Out[{\color{outcolor}11}]:} <matplotlib.legend.Legend at 0x1191a0a20>
\end{Verbatim}
            
    \begin{center}
    \adjustimage{max size={0.9\linewidth}{0.9\paperheight}}{output_10_1.png}
    \end{center}
    { \hspace*{\fill} \\}
    
    For predicting, sometimes my implementation is better and sometimes
worse.

    \begin{Verbatim}[commandchars=\\\{\}]
{\color{incolor}In [{\color{incolor}12}]:} \PY{l+s+sd}{\PYZdq{}\PYZdq{}\PYZdq{}}
         \PY{l+s+sd}{For the next tests,we now try with bigger forests, }
         \PY{l+s+sd}{we start with 25 trees and keep building forests each time by }
         \PY{l+s+sd}{incrementing the number of trees per forest by 25 until we reach 500 }
         \PY{l+s+sd}{trees in a forest. Each forest was max\PYZus{}depth 10.}
         \PY{l+s+sd}{\PYZdq{}\PYZdq{}\PYZdq{}}
         
         \PY{n}{scikit\PYZus{}build\PYZus{}time\PYZus{}2} \PY{o}{=} \PY{n}{np}\PY{o}{.}\PY{n}{array}\PY{p}{(}\PY{p}{[}\PY{p}{(}\PY{l+m+mi}{25}\PY{p}{,} \PY{l+m+mf}{0.14169907569885254}\PY{p}{)}\PY{p}{,} \PY{p}{(}\PY{l+m+mi}{50}\PY{p}{,} \PY{l+m+mf}{0.16149616241455078}\PY{p}{)}\PY{p}{,} \PY{p}{(}\PY{l+m+mi}{75}\PY{p}{,} \PY{l+m+mf}{0.2878692150115967}\PY{p}{)}\PY{p}{,} \PY{p}{(}\PY{l+m+mi}{100}\PY{p}{,} \PY{l+m+mf}{0.32459115982055664}\PY{p}{)}\PY{p}{,} \PY{p}{(}\PY{l+m+mi}{125}\PY{p}{,} \PY{l+m+mf}{0.3455038070678711}\PY{p}{)}\PY{p}{,} \PY{p}{(}\PY{l+m+mi}{150}\PY{p}{,} \PY{l+m+mf}{0.47414708137512207}\PY{p}{)}\PY{p}{,} \PY{p}{(}\PY{l+m+mi}{175}\PY{p}{,} \PY{l+m+mf}{0.50897216796875}\PY{p}{)}\PY{p}{,} \PY{p}{(}\PY{l+m+mi}{200}\PY{p}{,} \PY{l+m+mf}{0.6480038166046143}\PY{p}{)}\PY{p}{,} \PY{p}{(}\PY{l+m+mi}{225}\PY{p}{,} \PY{l+m+mf}{0.7711877822875977}\PY{p}{)}\PY{p}{,} \PY{p}{(}\PY{l+m+mi}{250}\PY{p}{,} \PY{l+m+mf}{0.7430751323699951}\PY{p}{)}\PY{p}{,} \PY{p}{(}\PY{l+m+mi}{275}\PY{p}{,} \PY{l+m+mf}{0.9384067058563232}\PY{p}{)}\PY{p}{,} \PY{p}{(}\PY{l+m+mi}{300}\PY{p}{,} \PY{l+m+mf}{0.8537948131561279}\PY{p}{)}\PY{p}{,} \PY{p}{(}\PY{l+m+mi}{325}\PY{p}{,} \PY{l+m+mf}{0.9658739566802979}\PY{p}{)}\PY{p}{,} \PY{p}{(}\PY{l+m+mi}{350}\PY{p}{,} \PY{l+m+mf}{1.2162902355194092}\PY{p}{)}\PY{p}{,} \PY{p}{(}\PY{l+m+mi}{375}\PY{p}{,} \PY{l+m+mf}{1.3096709251403809}\PY{p}{)}\PY{p}{,} \PY{p}{(}\PY{l+m+mi}{400}\PY{p}{,} \PY{l+m+mf}{1.2872178554534912}\PY{p}{)}\PY{p}{,} \PY{p}{(}\PY{l+m+mi}{425}\PY{p}{,} \PY{l+m+mf}{1.2292540073394775}\PY{p}{)}\PY{p}{,} \PY{p}{(}\PY{l+m+mi}{450}\PY{p}{,} \PY{l+m+mf}{1.311039924621582}\PY{p}{)}\PY{p}{,} \PY{p}{(}\PY{l+m+mi}{475}\PY{p}{,} \PY{l+m+mf}{1.35695219039917}\PY{p}{)}\PY{p}{,} \PY{p}{(}\PY{l+m+mi}{500}\PY{p}{,} \PY{l+m+mf}{1.4725921154022217}\PY{p}{)}\PY{p}{]}\PY{p}{)}
         \PY{n}{my\PYZus{}build\PYZus{}time\PYZus{}2} \PY{o}{=} \PY{n}{np}\PY{o}{.}\PY{n}{array}\PY{p}{(}\PY{p}{[}\PY{p}{(}\PY{l+m+mi}{25}\PY{p}{,} \PY{l+m+mf}{15.01}\PY{p}{)}\PY{p}{,} \PY{p}{(}\PY{l+m+mi}{50}\PY{p}{,} \PY{l+m+mf}{29.608}\PY{p}{)}\PY{p}{,} \PY{p}{(}\PY{l+m+mi}{75}\PY{p}{,} \PY{l+m+mf}{42.202}\PY{p}{)}\PY{p}{,} \PY{p}{(}\PY{l+m+mi}{100}\PY{p}{,} \PY{l+m+mf}{55.156}\PY{p}{)}\PY{p}{,} \PY{p}{(}\PY{l+m+mi}{125}\PY{p}{,} \PY{l+m+mf}{74.757}\PY{p}{)}\PY{p}{,} \PY{p}{(}\PY{l+m+mi}{150}\PY{p}{,} \PY{l+m+mf}{94.441}\PY{p}{)}\PY{p}{,} \PY{p}{(}\PY{l+m+mi}{175}\PY{p}{,} \PY{l+m+mf}{106.09}\PY{p}{)}\PY{p}{,} \PY{p}{(}\PY{l+m+mi}{200}\PY{p}{,} \PY{l+m+mf}{120.511}\PY{p}{)}\PY{p}{,} \PY{p}{(}\PY{l+m+mi}{225}\PY{p}{,} \PY{l+m+mf}{132.117}\PY{p}{)}\PY{p}{,} \PY{p}{(}\PY{l+m+mi}{250}\PY{p}{,} \PY{l+m+mf}{158.548}\PY{p}{)}\PY{p}{,} \PY{p}{(}\PY{l+m+mi}{275}\PY{p}{,} \PY{l+m+mf}{172.363}\PY{p}{)}\PY{p}{,} \PY{p}{(}\PY{l+m+mi}{300}\PY{p}{,} \PY{l+m+mf}{186.447}\PY{p}{)}\PY{p}{,} \PY{p}{(}\PY{l+m+mi}{325}\PY{p}{,} \PY{l+m+mf}{189.227}\PY{p}{)}\PY{p}{,} \PY{p}{(}\PY{l+m+mi}{350}\PY{p}{,} \PY{l+m+mf}{209.752}\PY{p}{)}\PY{p}{,} \PY{p}{(}\PY{l+m+mi}{375}\PY{p}{,} \PY{l+m+mf}{199.446}\PY{p}{)}\PY{p}{,} \PY{p}{(}\PY{l+m+mi}{400}\PY{p}{,} \PY{l+m+mf}{216.24}\PY{p}{)}\PY{p}{,} \PY{p}{(}\PY{l+m+mi}{425}\PY{p}{,} \PY{l+m+mf}{260.016}\PY{p}{)}\PY{p}{,} \PY{p}{(}\PY{l+m+mi}{450}\PY{p}{,} \PY{l+m+mf}{264.683}\PY{p}{)}\PY{p}{,} \PY{p}{(}\PY{l+m+mi}{475}\PY{p}{,} \PY{l+m+mf}{278.791}\PY{p}{)}\PY{p}{,} \PY{p}{(}\PY{l+m+mi}{500}\PY{p}{,} \PY{l+m+mf}{299.935}\PY{p}{)}\PY{p}{]}\PY{p}{)}
         
         \PY{c+c1}{\PYZsh{} results for predicting forest with the same trees }
         \PY{n}{sckikit\PYZus{}predict\PYZus{}time\PYZus{}2} \PY{o}{=} \PY{n}{np}\PY{o}{.}\PY{n}{array}\PY{p}{(}\PY{p}{[}\PY{p}{(}\PY{l+m+mi}{25}\PY{p}{,} \PY{l+m+mf}{0.10662674903869629}\PY{p}{)}\PY{p}{,} \PY{p}{(}\PY{l+m+mi}{50}\PY{p}{,} \PY{l+m+mf}{0.10692405700683594}\PY{p}{)}\PY{p}{,} \PY{p}{(}\PY{l+m+mi}{75}\PY{p}{,} \PY{l+m+mf}{0.10454201698303223}\PY{p}{)}\PY{p}{,} \PY{p}{(}\PY{l+m+mi}{100}\PY{p}{,} \PY{l+m+mf}{0.10660600662231445}\PY{p}{)}\PY{p}{,} \PY{p}{(}\PY{l+m+mi}{125}\PY{p}{,} \PY{l+m+mf}{0.10626888275146484}\PY{p}{)}\PY{p}{,} \PY{p}{(}\PY{l+m+mi}{150}\PY{p}{,} \PY{l+m+mf}{0.1060628890991211}\PY{p}{)}\PY{p}{,} \PY{p}{(}\PY{l+m+mi}{175}\PY{p}{,} \PY{l+m+mf}{0.10494303703308105}\PY{p}{)}\PY{p}{,} \PY{p}{(}\PY{l+m+mi}{200}\PY{p}{,} \PY{l+m+mf}{0.1116788387298584}\PY{p}{)}\PY{p}{,} \PY{p}{(}\PY{l+m+mi}{225}\PY{p}{,} \PY{l+m+mf}{0.10396790504455566}\PY{p}{)}\PY{p}{,} \PY{p}{(}\PY{l+m+mi}{250}\PY{p}{,} \PY{l+m+mf}{0.10534286499023438}\PY{p}{)}\PY{p}{,} \PY{p}{(}\PY{l+m+mi}{275}\PY{p}{,} \PY{l+m+mf}{0.10752487182617188}\PY{p}{)}\PY{p}{,} \PY{p}{(}\PY{l+m+mi}{300}\PY{p}{,} \PY{l+m+mf}{0.21181225776672363}\PY{p}{)}\PY{p}{,} \PY{p}{(}\PY{l+m+mi}{325}\PY{p}{,} \PY{l+m+mf}{0.21249985694885254}\PY{p}{)}\PY{p}{,} \PY{p}{(}\PY{l+m+mi}{350}\PY{p}{,} \PY{l+m+mf}{0.20537304878234863}\PY{p}{)}\PY{p}{,} \PY{p}{(}\PY{l+m+mi}{375}\PY{p}{,} \PY{l+m+mf}{0.20955109596252441}\PY{p}{)}\PY{p}{,} \PY{p}{(}\PY{l+m+mi}{400}\PY{p}{,} \PY{l+m+mf}{0.2091829776763916}\PY{p}{)}\PY{p}{,} \PY{p}{(}\PY{l+m+mi}{425}\PY{p}{,} \PY{l+m+mf}{0.20935487747192383}\PY{p}{)}\PY{p}{,} \PY{p}{(}\PY{l+m+mi}{450}\PY{p}{,} \PY{l+m+mf}{0.20930695533752441}\PY{p}{)}\PY{p}{,} \PY{p}{(}\PY{l+m+mi}{475}\PY{p}{,} \PY{l+m+mf}{0.20959877967834473}\PY{p}{)}\PY{p}{,} \PY{p}{(}\PY{l+m+mi}{500}\PY{p}{,} \PY{l+m+mf}{0.20958900451660156}\PY{p}{)}\PY{p}{]}\PY{p}{)}
         \PY{n}{my\PYZus{}predict\PYZus{}time\PYZus{}2} \PY{o}{=} \PY{n}{np}\PY{o}{.}\PY{n}{array}\PY{p}{(}\PY{p}{[}\PY{p}{(}\PY{l+m+mi}{25}\PY{p}{,} \PY{l+m+mf}{0.08}\PY{p}{)}\PY{p}{,} \PY{p}{(}\PY{l+m+mi}{50}\PY{p}{,} \PY{l+m+mf}{0.086}\PY{p}{)}\PY{p}{,} \PY{p}{(}\PY{l+m+mi}{75}\PY{p}{,} \PY{l+m+mf}{0.095}\PY{p}{)}\PY{p}{,} \PY{p}{(}\PY{l+m+mi}{100}\PY{p}{,} \PY{l+m+mf}{0.092}\PY{p}{)}\PY{p}{,} \PY{p}{(}\PY{l+m+mi}{125}\PY{p}{,} \PY{l+m+mf}{0.101}\PY{p}{)}\PY{p}{,} \PY{p}{(}\PY{l+m+mi}{150}\PY{p}{,} \PY{l+m+mf}{0.094}\PY{p}{)}\PY{p}{,} \PY{p}{(}\PY{l+m+mi}{175}\PY{p}{,} \PY{l+m+mf}{0.109}\PY{p}{)}\PY{p}{,} \PY{p}{(}\PY{l+m+mi}{200}\PY{p}{,} \PY{l+m+mf}{0.112}\PY{p}{)}\PY{p}{,} \PY{p}{(}\PY{l+m+mi}{225}\PY{p}{,} \PY{l+m+mf}{0.116}\PY{p}{)}\PY{p}{,} \PY{p}{(}\PY{l+m+mi}{250}\PY{p}{,} \PY{l+m+mf}{0.157}\PY{p}{)}\PY{p}{,} \PY{p}{(}\PY{l+m+mi}{275}\PY{p}{,} \PY{l+m+mf}{0.14}\PY{p}{)}\PY{p}{,} \PY{p}{(}\PY{l+m+mi}{300}\PY{p}{,} \PY{l+m+mf}{0.137}\PY{p}{)}\PY{p}{,} \PY{p}{(}\PY{l+m+mi}{325}\PY{p}{,} \PY{l+m+mf}{0.132}\PY{p}{)}\PY{p}{,} \PY{p}{(}\PY{l+m+mi}{350}\PY{p}{,} \PY{l+m+mf}{0.141}\PY{p}{)}\PY{p}{,} \PY{p}{(}\PY{l+m+mi}{375}\PY{p}{,} \PY{l+m+mf}{0.148}\PY{p}{)}\PY{p}{,} \PY{p}{(}\PY{l+m+mi}{400}\PY{p}{,} \PY{l+m+mf}{0.148}\PY{p}{)}\PY{p}{,} \PY{p}{(}\PY{l+m+mi}{425}\PY{p}{,} \PY{l+m+mf}{0.254}\PY{p}{)}\PY{p}{,} \PY{p}{(}\PY{l+m+mi}{450}\PY{p}{,} \PY{l+m+mf}{0.156}\PY{p}{)}\PY{p}{,} \PY{p}{(}\PY{l+m+mi}{475}\PY{p}{,} \PY{l+m+mf}{0.199}\PY{p}{)}\PY{p}{,} \PY{p}{(}\PY{l+m+mi}{500}\PY{p}{,} \PY{l+m+mf}{0.192}\PY{p}{)}\PY{p}{]}\PY{p}{)}
         
         \PY{c+c1}{\PYZsh{} score from previous trees}
         \PY{n}{sckikit\PYZus{}score\PYZus{}2} \PY{o}{=} \PY{n}{np}\PY{o}{.}\PY{n}{array}\PY{p}{(}\PY{p}{[}\PY{p}{(}\PY{l+m+mi}{25}\PY{p}{,} \PY{l+m+mf}{0.835}\PY{p}{)}\PY{p}{,} \PY{p}{(}\PY{l+m+mi}{50}\PY{p}{,} \PY{l+m+mf}{0.8666666666666667}\PY{p}{)}\PY{p}{,} \PY{p}{(}\PY{l+m+mi}{75}\PY{p}{,} \PY{l+m+mf}{0.8783333333333333}\PY{p}{)}\PY{p}{,} \PY{p}{(}\PY{l+m+mi}{100}\PY{p}{,} \PY{l+m+mf}{0.8716666666666667}\PY{p}{)}\PY{p}{,} \PY{p}{(}\PY{l+m+mi}{125}\PY{p}{,} \PY{l+m+mf}{0.88}\PY{p}{)}\PY{p}{,} \PY{p}{(}\PY{l+m+mi}{150}\PY{p}{,} \PY{l+m+mf}{0.865}\PY{p}{)}\PY{p}{,} \PY{p}{(}\PY{l+m+mi}{175}\PY{p}{,} \PY{l+m+mf}{0.8666666666666667}\PY{p}{)}\PY{p}{,} \PY{p}{(}\PY{l+m+mi}{200}\PY{p}{,} \PY{l+m+mf}{0.8683333333333333}\PY{p}{)}\PY{p}{,} \PY{p}{(}\PY{l+m+mi}{225}\PY{p}{,} \PY{l+m+mf}{0.885}\PY{p}{)}\PY{p}{,} \PY{p}{(}\PY{l+m+mi}{250}\PY{p}{,} \PY{l+m+mf}{0.87}\PY{p}{)}\PY{p}{,} \PY{p}{(}\PY{l+m+mi}{275}\PY{p}{,} \PY{l+m+mf}{0.88}\PY{p}{)}\PY{p}{,} \PY{p}{(}\PY{l+m+mi}{300}\PY{p}{,} \PY{l+m+mf}{0.8783333333333333}\PY{p}{)}\PY{p}{,} \PY{p}{(}\PY{l+m+mi}{325}\PY{p}{,} \PY{l+m+mf}{0.88}\PY{p}{)}\PY{p}{,} \PY{p}{(}\PY{l+m+mi}{350}\PY{p}{,} \PY{l+m+mf}{0.8733333333333333}\PY{p}{)}\PY{p}{,} \PY{p}{(}\PY{l+m+mi}{375}\PY{p}{,} \PY{l+m+mf}{0.885}\PY{p}{)}\PY{p}{,} \PY{p}{(}\PY{l+m+mi}{400}\PY{p}{,} \PY{l+m+mf}{0.88}\PY{p}{)}\PY{p}{,} \PY{p}{(}\PY{l+m+mi}{425}\PY{p}{,} \PY{l+m+mf}{0.87}\PY{p}{)}\PY{p}{,} \PY{p}{(}\PY{l+m+mi}{450}\PY{p}{,} \PY{l+m+mf}{0.8783333333333333}\PY{p}{)}\PY{p}{,} \PY{p}{(}\PY{l+m+mi}{475}\PY{p}{,} \PY{l+m+mf}{0.8766666666666667}\PY{p}{)}\PY{p}{,} \PY{p}{(}\PY{l+m+mi}{500}\PY{p}{,} \PY{l+m+mf}{0.8766666666666667}\PY{p}{)}\PY{p}{]}\PY{p}{)}
         \PY{n}{my\PYZus{}score\PYZus{}2} \PY{o}{=} \PY{n}{np}\PY{o}{.}\PY{n}{array}\PY{p}{(}\PY{p}{[}\PY{p}{(}\PY{l+m+mi}{25}\PY{p}{,} \PY{l+m+mf}{0.805}\PY{p}{)}\PY{p}{,} \PY{p}{(}\PY{l+m+mi}{50}\PY{p}{,} \PY{l+m+mf}{0.821667}\PY{p}{)}\PY{p}{,} \PY{p}{(}\PY{l+m+mi}{50}\PY{p}{,} \PY{l+m+mf}{0.825}\PY{p}{)}\PY{p}{,} \PY{p}{(}\PY{l+m+mi}{100}\PY{p}{,} \PY{l+m+mf}{0.743333}\PY{p}{)}\PY{p}{,} \PY{p}{(}\PY{l+m+mi}{125}\PY{p}{,} \PY{l+m+mf}{0.825}\PY{p}{)}\PY{p}{,} \PY{p}{(}\PY{l+m+mi}{150}\PY{p}{,} \PY{l+m+mf}{0.661667}\PY{p}{)}\PY{p}{,} \PY{p}{(}\PY{l+m+mi}{175}\PY{p}{,} \PY{l+m+mf}{0.816667}\PY{p}{)}\PY{p}{,} \PY{p}{(}\PY{l+m+mi}{200}\PY{p}{,} \PY{l+m+mf}{0.731667}\PY{p}{)}\PY{p}{,} \PY{p}{(}\PY{l+m+mi}{225}\PY{p}{,} \PY{l+m+mf}{0.816667}\PY{p}{)}\PY{p}{,} \PY{p}{(}\PY{l+m+mi}{250}\PY{p}{,} \PY{l+m+mf}{0.815}\PY{p}{)}\PY{p}{,} \PY{p}{(}\PY{l+m+mi}{275}\PY{p}{,} \PY{l+m+mf}{0.8}\PY{p}{)}\PY{p}{,} \PY{p}{(}\PY{l+m+mi}{300}\PY{p}{,} \PY{l+m+mf}{0.826667}\PY{p}{)}\PY{p}{,} \PY{p}{(}\PY{l+m+mi}{325}\PY{p}{,} \PY{l+m+mf}{0.745}\PY{p}{)}\PY{p}{,} \PY{p}{(}\PY{l+m+mi}{350}\PY{p}{,} \PY{l+m+mf}{0.568333}\PY{p}{)}\PY{p}{,} \PY{p}{(}\PY{l+m+mi}{375}\PY{p}{,} \PY{l+m+mf}{0.808333}\PY{p}{)}\PY{p}{,} \PY{p}{(}\PY{l+m+mi}{400}\PY{p}{,} \PY{l+m+mf}{0.816667}\PY{p}{)}\PY{p}{,} \PY{p}{(}\PY{l+m+mi}{425}\PY{p}{,} \PY{l+m+mf}{0.83}\PY{p}{)}\PY{p}{,} \PY{p}{(}\PY{l+m+mi}{450}\PY{p}{,} \PY{l+m+mf}{0.82}\PY{p}{)}\PY{p}{,} \PY{p}{(}\PY{l+m+mi}{475}\PY{p}{,} \PY{l+m+mf}{0.743333}\PY{p}{)}\PY{p}{,} \PY{p}{(}\PY{l+m+mi}{500}\PY{p}{,} \PY{l+m+mf}{0.72}\PY{p}{)}\PY{p}{]}\PY{p}{)}
\end{Verbatim}


    \begin{Verbatim}[commandchars=\\\{\}]
{\color{incolor}In [{\color{incolor}13}]:} \PY{n}{plt}\PY{o}{.}\PY{n}{ylabel}\PY{p}{(}\PY{l+s+s1}{\PYZsq{}}\PY{l+s+s1}{Time in seconds}\PY{l+s+s1}{\PYZsq{}}\PY{p}{)}
         \PY{n}{plt}\PY{o}{.}\PY{n}{xlabel}\PY{p}{(}\PY{l+s+s1}{\PYZsq{}}\PY{l+s+s1}{Number of trees}\PY{l+s+s1}{\PYZsq{}}\PY{p}{)}
         \PY{n}{plt}\PY{o}{.}\PY{n}{title}\PY{p}{(}\PY{l+s+s1}{\PYZsq{}}\PY{l+s+s1}{Time building forest with max\PYZhy{}depth 10}\PY{l+s+s1}{\PYZsq{}}\PY{p}{)}
         \PY{n}{plt}\PY{o}{.}\PY{n}{plot}\PY{p}{(}\PY{o}{*}\PY{n+nb}{zip}\PY{p}{(}\PY{o}{*}\PY{n}{scikit\PYZus{}build\PYZus{}time\PYZus{}2}\PY{p}{)}\PY{p}{,} \PY{n}{label}\PY{o}{=}\PY{l+s+s2}{\PYZdq{}}\PY{l+s+s2}{scikit\PYZhy{}learn}\PY{l+s+s2}{\PYZdq{}}\PY{p}{)}
         \PY{n}{plt}\PY{o}{.}\PY{n}{plot}\PY{p}{(}\PY{o}{*}\PY{n+nb}{zip}\PY{p}{(}\PY{o}{*}\PY{n}{my\PYZus{}build\PYZus{}time\PYZus{}2}\PY{p}{)}\PY{p}{,} \PY{n}{label}\PY{o}{=}\PY{l+s+s2}{\PYZdq{}}\PY{l+s+s2}{my\PYZus{}implementation}\PY{l+s+s2}{\PYZdq{}}\PY{p}{)}
         \PY{n}{plt}\PY{o}{.}\PY{n}{legend}\PY{p}{(}\PY{p}{)}
\end{Verbatim}


\begin{Verbatim}[commandchars=\\\{\}]
{\color{outcolor}Out[{\color{outcolor}13}]:} <matplotlib.legend.Legend at 0x1193a8be0>
\end{Verbatim}
            
    \begin{center}
    \adjustimage{max size={0.9\linewidth}{0.9\paperheight}}{output_13_1.png}
    \end{center}
    { \hspace*{\fill} \\}
    
    Now, with bigger forests we can see how almost exponentially slower is
my implementation. Scikit-learn looks like it always takes the same
time.

    \begin{Verbatim}[commandchars=\\\{\}]
{\color{incolor}In [{\color{incolor}14}]:} \PY{n}{plt}\PY{o}{.}\PY{n}{ylabel}\PY{p}{(}\PY{l+s+s1}{\PYZsq{}}\PY{l+s+s1}{Time in seconds}\PY{l+s+s1}{\PYZsq{}}\PY{p}{)}
         \PY{n}{plt}\PY{o}{.}\PY{n}{xlabel}\PY{p}{(}\PY{l+s+s1}{\PYZsq{}}\PY{l+s+s1}{Number of trees}\PY{l+s+s1}{\PYZsq{}}\PY{p}{)}
         \PY{n}{plt}\PY{o}{.}\PY{n}{title}\PY{p}{(}\PY{l+s+s1}{\PYZsq{}}\PY{l+s+s1}{Time predicting forest with max\PYZhy{}depth 10}\PY{l+s+s1}{\PYZsq{}}\PY{p}{)}
         \PY{n}{plt}\PY{o}{.}\PY{n}{plot}\PY{p}{(}\PY{o}{*}\PY{n+nb}{zip}\PY{p}{(}\PY{o}{*}\PY{n}{sckikit\PYZus{}predict\PYZus{}time\PYZus{}2}\PY{p}{)}\PY{p}{,} \PY{n}{label}\PY{o}{=}\PY{l+s+s2}{\PYZdq{}}\PY{l+s+s2}{scikit\PYZhy{}learn}\PY{l+s+s2}{\PYZdq{}}\PY{p}{)}
         \PY{n}{plt}\PY{o}{.}\PY{n}{plot}\PY{p}{(}\PY{o}{*}\PY{n+nb}{zip}\PY{p}{(}\PY{o}{*}\PY{n}{my\PYZus{}predict\PYZus{}time\PYZus{}2}\PY{p}{)}\PY{p}{,} \PY{n}{label}\PY{o}{=}\PY{l+s+s2}{\PYZdq{}}\PY{l+s+s2}{my\PYZus{}implementation}\PY{l+s+s2}{\PYZdq{}}\PY{p}{)}
         \PY{n}{plt}\PY{o}{.}\PY{n}{legend}\PY{p}{(}\PY{p}{)}
\end{Verbatim}


\begin{Verbatim}[commandchars=\\\{\}]
{\color{outcolor}Out[{\color{outcolor}14}]:} <matplotlib.legend.Legend at 0x1193fbd30>
\end{Verbatim}
            
    \begin{center}
    \adjustimage{max size={0.9\linewidth}{0.9\paperheight}}{output_15_1.png}
    \end{center}
    { \hspace*{\fill} \\}
    
    For predicting big forests, my implementation seems to outperform the
other one when the size of the forest reaches 200.

    \begin{Verbatim}[commandchars=\\\{\}]
{\color{incolor}In [{\color{incolor}15}]:} \PY{n}{plt}\PY{o}{.}\PY{n}{ylabel}\PY{p}{(}\PY{l+s+s1}{\PYZsq{}}\PY{l+s+s1}{Time in seconds}\PY{l+s+s1}{\PYZsq{}}\PY{p}{)}
         \PY{n}{plt}\PY{o}{.}\PY{n}{xlabel}\PY{p}{(}\PY{l+s+s1}{\PYZsq{}}\PY{l+s+s1}{Number of trees}\PY{l+s+s1}{\PYZsq{}}\PY{p}{)}
         \PY{n}{plt}\PY{o}{.}\PY{n}{title}\PY{p}{(}\PY{l+s+s1}{\PYZsq{}}\PY{l+s+s1}{Score from predicting forest with max depth 10}\PY{l+s+s1}{\PYZsq{}}\PY{p}{)}
         \PY{n}{plt}\PY{o}{.}\PY{n}{plot}\PY{p}{(}\PY{o}{*}\PY{n+nb}{zip}\PY{p}{(}\PY{o}{*}\PY{n}{sckikit\PYZus{}score\PYZus{}2}\PY{p}{)}\PY{p}{,} \PY{n}{label}\PY{o}{=}\PY{l+s+s2}{\PYZdq{}}\PY{l+s+s2}{scikit\PYZhy{}learn}\PY{l+s+s2}{\PYZdq{}}\PY{p}{)}
         \PY{n}{plt}\PY{o}{.}\PY{n}{plot}\PY{p}{(}\PY{o}{*}\PY{n+nb}{zip}\PY{p}{(}\PY{o}{*}\PY{n}{my\PYZus{}score\PYZus{}2}\PY{p}{)}\PY{p}{,} \PY{n}{label}\PY{o}{=}\PY{l+s+s2}{\PYZdq{}}\PY{l+s+s2}{my\PYZus{}implementation}\PY{l+s+s2}{\PYZdq{}}\PY{p}{)}
         \PY{n}{plt}\PY{o}{.}\PY{n}{legend}\PY{p}{(}\PY{p}{)}
\end{Verbatim}


\begin{Verbatim}[commandchars=\\\{\}]
{\color{outcolor}Out[{\color{outcolor}15}]:} <matplotlib.legend.Legend at 0x119774668>
\end{Verbatim}
            
    \begin{center}
    \adjustimage{max size={0.9\linewidth}{0.9\paperheight}}{output_17_1.png}
    \end{center}
    { \hspace*{\fill} \\}
    
    For predicting, my implementation is doing terrible compared to the one
from scikit-learn. Also, it is not stable, it goes up and down by a lot
while the other one keeps within the same range of score.

    \paragraph{Conclusion}\label{conclusion}

I am dissapointed on how bad I did against scikit-learn implementation.
I naively though that I could outperform it or at least have similar
benchmarks. It is true that the one form scikit-learn is optimized and
has been refactorized by a lot of developers and optimized to work fast.
During the summer I still want to keep working on this project and try
to at least take the times closer to scikit-learn's Random Forest
Classifier, as well as:

\begin{itemize}
\item
  As I already mentioned, change the way in how the splitting attribute
  is chosen, right now the complexity for that is O(n**2), this is why
  it is so slow. There is a
  \href{https://stats.stackexchange.com/questions/105487/the-efficiency-of-decision-tree}{way}
  in which we can make it O(n), which will reduce the computing time for
  the splitting criterion reduce a lot.
\item
  Implement the functionality of telling which features are the most
  important. This is a key property of Random Forests.
\item
  Use new C++11 smart pointers instead of old-pointers, code will look
  cleaner and memory management will be optimized, as well as improve
  performance.
\item
  Add functionality to support missing values form the dataset.
\item
  Create exceptions when calling the classes so that it can work like
  the implementation from
  \href{http://scikit-learn.org/stable/modules/generated/sklearn.ensemble.RandomForestClassifier.html\#sklearn.ensemble.RandomForestClassifier.predict}{scikit-learn}
\item
  Document the whole project better so in the future people can actually
  use it.
\end{itemize}

    \paragraph{References}\label{references}

    \begin{itemize}
\tightlist
\item
  https://stats.stackexchange.com/questions/105487/the-efficiency-of-decision-tree
\item
  https://github.com/scikit-learn/scikit-learn/blob/a24c8b46/sklearn/ensemble/forest.py\#L518
\item
  https://www.stat.berkeley.edu/users/breiman/randomforest2001.pdf
\item
  https://web.stanford.edu/\textasciitilde{}hastie/Papers/ESLII.pdf
\item
  https://quantdare.com/what-is-the-difference-between-bagging-and-boosting/
\item
  http://ect.bell-labs.com/who/tkh/publications/papers/odt.pdf
\item
  http://blog.citizennet.com/blog/2012/11/10/random-forests-ensembles-and-performance-metrics
\item
  https://www.stat.berkeley.edu/\textasciitilde{}breiman/RandomForests/cc\_home.htm
\item
  https://cs.brown.edu/courses/cs123/docs/helpsessions/intermediate.pdf
\item
  https://github.com/scikit-learn/scikit-learn/issues/1435
\end{itemize}


    % Add a bibliography block to the postdoc
    
    
    
    \end{document}
